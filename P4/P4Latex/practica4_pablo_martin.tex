\documentclass{article}

\usepackage[utf8]{inputenc}
\usepackage{graphicx}
\usepackage{xcolor, colortbl}
\usepackage{tabularx} % Para tablas con ancho automático
\usepackage{adjustbox} % Para ajustar la tabla
\usepackage[margin=2cm]{geometry} % Modifica los márgenes de la páginaç
\usepackage{hyperref}





\begin{document}

\begin{titlepage}
    \centering
    
    \vspace{1cm}
    {\scshape\LARGE Universidad de Valladolid \par}
    \vspace{1.5cm}
    {\huge\bfseries Práctica 4 - TAA\par}
    \vspace{0.5cm}
    {\Large\itshape Pablo Martín de Benito\par}
    \vspace{1.5cm}
    \includegraphics[width=0.3\textwidth]{logo-universidad-de-valladolid.png}\par

    \vfill
    \vfill
    {\large \today\par}
\end{titlepage}

\tableofcontents % Genera el índice

\newpage % Salto de página antes del contenido del documento

\section{Método 50T, Resto}

\begin{table}[h]
	\begin{adjustbox}{center}
		\begin{tabular}{|c|c|c|c|c|c|}
			\hline
			\rowcolor[gray]{0.8}
			\textbf{Datos} & \textbf{Algoritmo} & \multicolumn{4}{|c|}{\textbf{Método 50T, resto}} \\ \hline
			\rowcolor[gray]{0.8}			
			\multicolumn{2}{|c|}{\textbf{ }} & \textbf{Tasa Error} & \textbf{Desviación Estándar} & 			\multicolumn{2}{|c|}{\textbf{Intervalo}} \\ \hline
			
			soybean & j48 & 0,411671924290221 & 0,0195606623587688 & 0,373333026067034 & 	0,450010822513408 \\ \hline
			 & sin podar	 & 0,416403785488959 & 0,0195934860745905 & 0,378000552782762 & 0,454807018195156 \\ \hline
			Vote	 & j48 & 0,077720207253886 & 0,0136448245874235 & 0,050976351062536 & 0,104464063445236 \\ \hline
			 & sin podar	 & 0,077720207253886 & 0,0136448245874235 & 0,050976351062536 & 	0,104464063445236 \\ \hline
		\end{tabular}
	\end{adjustbox}
\end{table}

Para ello, hemos utilizado los siguientes porcentajes de entrenamiento y clasificación:
\begin{itemize}
	\item 7,32064421669107 \% para soybean
	\item 11,4942528735632 \% para vote
\end{itemize}
\vspace{0.5cm}

Además hemos utilizado las fórmulas de desviación típica e intervalo normales.
\begin{itemize}
	\item Desviación Estándar:	 $S_{e(h)} = (\frac{e_{S(h)}(1-e_{S(h)}}{n})^{1/2}$
	\item Intervalo de Confianza Normal:	$[e_{S(h)}\pm Z_{1-\alpha}\quad S_{e(h)}]$
\end{itemize}

\vspace{1cm}



\section{Método Hold Out 2/3 - 1/3}
\begin{table}[h]
	\begin{adjustbox}{center}
		\begin{tabular}{|c|c|c|c|c|c|}
			\hline
			\rowcolor[gray]{0.8}
			\textbf{Datos} & \textbf{Algoritmo} & \multicolumn{4}{|c|}{\textbf{Hold Out}} \\ \hline
			\rowcolor[gray]{0.8}			
			\multicolumn{2}{|c|}{\textbf{ }} & \textbf{Tasa Error} & \textbf{Desviación Estándar} & 			\multicolumn{2}{|c|}{\textbf{Intervalo}} \\ \hline
			
			soybean & j48 & 0,103004291845494 & 0,012081503606538 & 0,0793245447766795 & 0,126684038914308\\ \hline
			 & sin podar	 & 0,107296137339056 & 0,0123010984913333 & 0,0831859842960427 & 0,131406290382069\\ \hline
			Vote	 & j48 & 0,0337837837837838 & 0,00920790734958442 & 0,0157362853785983 & 0,0518312821889693\\ \hline
			 & sin podar	 & 0,0337837837837838 & 0,00920790734958442 & 0,0157362853785983 & 0,0518312821889693\\ \hline
		\end{tabular}
	\end{adjustbox}
\end{table}

Utilizando las siguientes fórmulas para rellenar la tabla.
\begin{itemize}
	\item Desviación Estándar:	 $S_{e(h)} = (\frac{e_{S(h)}(1-e_{S(h)}}{n})^{1/2}$
	\item Intervalo de Confianza Normal:	$[e_{S(h)}\pm Z_{1-\alpha}\quad S_{e(h)}]$
\end{itemize}
 
\newpage

\section{Método Hold Out Repetido}
Realizamos tres experimentos más de Hold Out 2/3-1/3 y anotando la tasa de error en cada experimento.

\hspace{0.5cm} Utilizando tres semillas diferentes.\\

\begin{table}[h]
	\begin{adjustbox}{center}
	\begin{tabular}{|c|c|c|c|}
		\hline
		\rowcolor[gray]{0.8}
		\multicolumn{4}{|c|}{\textbf{Porcentaje}}\\ \hline
		
		\rowcolor[gray]{0.8}
		soybean & 2 & 3 & 4 \\ \hline
		
		j48 & 9,78723404255319 & 11,965811965812 & 11,6883116883117 \\ \hline 
		
		sin podar & 18,93617021276596 & 11,965811965812 & 13,8528138528139 \\ \hline

		\rowcolor[gray]{0.8}
		vote & 2 & 3 & 4 \\ \hline	
		
		j48 & 6,75675675675676 & 5,40540540540541 & 6,12244897959184 \\ \hline
		
		sin podar & 5,40540540540541 & 5,40540540540541 & 6,12244897959184\\ \hline
	\end{tabular}
	\end{adjustbox}
\end{table}

\vspace{1cm}

Ahora, con los cuatro experimentos, determinamos las tasas de error, desviación típica y los intervalos.


\begin{table}[h]
	\begin{adjustbox}{center}
		\begin{tabular}{|c|c|c|c|c|c|}
			\hline
			\rowcolor[gray]{0.8}
			\textbf{Datos} & \textbf{Algoritmo} & \multicolumn{4}{|c|}{\textbf{Hold Out Repetido}} \\ \hline
			\rowcolor[gray]{0.8}			
			\multicolumn{2}{|c|}{\textbf{ }} & \textbf{Tasa Error} & \textbf{Desviación Estándar} & 			\multicolumn{2}{|c|}{\textbf{Intervalo}} \\ \hline
			
			soybean & j48 & 0,109354467203066 & 0,0105673992752356 & 0,0925393784261278 & 0,126169555980004\\ \hline
			 & sin podar	 & 0,113711024413243 & 0,0206986099918583 & 0,0807749201261683 & 0,146647128700318\\ \hline
			Vote	 & j48 & 0,0541574738003309 & 0,0146614150119994 & 0,0308278930206922 & 0,0774870545799696\\ \hline
			 & sin podar	 & 0,0507790954219526 & 0,0118236710231369 & 0,0319649981455036 & 0,0695931926984016\\ \hline
		\end{tabular}
	\end{adjustbox}
\end{table}

Para obtener los resultados de la tabla, utilizamos las siguientes fórmulas.
\begin{itemize}
	\item Tasa Error: \hspace{0.5cm}	 $e(h) = [\sum_{i = 1,k}e_i(h)] / k$
	\item Desviación Estándar: \hspace{0.5cm}	$S_{e(h)} = \sqrt{\frac{1}{k-1} \quad \sum_{i=1,k}{(e_i(h)-e(h))²}}$
	\item Intervalo de Confianza T-Student: \hspace{0.5cm}	$[e(h) \pm t_{N,k-1} \quad S_{e(h)}/\sqrt{k}]$
	\item Siendo \hspace{0.5cm} qt(0.975,3) = 3,182446 \hspace{0.5cm} calculado con R.

\end{itemize}


\newpage

\section{Validación Cruzada 10 Particiones}
Se nos proporcionan los resultados de los experimentos de validación cruzada para los dos conjuntos

\begin{table}[h]
	\begin{adjustbox}{center}
		\begin{tabular}{|c|c|c|c|c|c|}
			\hline
			\rowcolor[gray]{0.8}
			\textbf{Datos} & \textbf{Algoritmo} & \multicolumn{4}{|c|}{\textbf{Validación Cruzada 10 Particiones}} \\ \hline
			\rowcolor[gray]{0.8}			
			\multicolumn{2}{|c|}{\textbf{ }} & \textbf{Tasa Error} & \textbf{Desviación Estándar} & 			\multicolumn{2}{|c|}{\textbf{Intervalo}} \\ \hline
			
			soybean & j48 & 0,0821500000000001 & 0,0106491783720623 & 0,0745320372568655 & 0,0897679627431346\\ \hline
			 & sin podar	 & 0,09224 & 0,00953312354081513 & 0,0854204147303841 & 0,0990595852696159\\ \hline
			Vote	 & j48 & 0,0342699999999999 & 0,0057205380477325 & 0,0301777738961873 & 0,0383622261038125\\ \hline
			 & sin podar	 & 0,0423699999999999 & 0,00881980725412975 & 0,0360606904254833 & 0,0486793095745165\\ \hline
		\end{tabular}
	\end{adjustbox}
\end{table}

Para obtener los resultados de la tabla, utilizamos las siguientes fórmulas.

\begin{itemize}
	\item Tasa Error: \hspace{0.5cm}	 $e(h) = [\sum_{i = 1,k}e_i(h)] / k $
	\item Desviación Estándar: \hspace{0.5cm}	$S_{e(h)} = \sqrt{\frac{1}{k-1} \quad \sum_{i=1,k}{(e_i(h)-e(h))^2}}$
	\item Intervalo de Confianza T-Student: \hspace{0.5cm}	$[e(h) \pm t_{N,k-1} \quad S_{e(h)}/\sqrt{k}]$
	\item Siendo \hspace{0.5cm} qt(0.975,9) = 2,262157 \hspace{0.5cm} calculado con R.
\end{itemize}
Siendo K = 10 Particiones

\vspace{1cm}


\section{Validación Cruzada Repetida}
Realizamos tres experimentos más de validación cruzada, anotando el error medio obtenido.

\hspace{0.5cm} Utilizando tres semillas diferentes para cada experimento.

\begin{table}[h]
	\begin{adjustbox}{center}
	\begin{tabular}{|c|c|c|c|}
		\hline
		\rowcolor[gray]{0.8}
		\multicolumn{4}{|c|}{\textbf{Porcentaje}}\\ \hline
		
		\rowcolor[gray]{0.8}
		soybean & 2 & 3 & 4 \\ \hline
		
		j48 & 9,78723404255319 & 11,965811965812 & 11,6883116883117 \\ \hline 
		
		sin podar & 18,93617021276596 & 11,965811965812 & 13,8528138528139 \\ \hline

		\rowcolor[gray]{0.8}
		vote & 2 & 3 & 4 \\ \hline	
		
		j48 & 6,75675675675676 & 5,40540540540541 & 6,12244897959184 \\ \hline
		
		sin podar & 5,40540540540541 & 5,40540540540541 & 6,12244897959184\\ \hline
	\end{tabular}
	\end{adjustbox}
\end{table}

Con los cuatro experimentos de validación cruzada repetida determinamos la tasa de error, la desviación estándar y el intervalo de confianza para cada conjunto de datos y algoritmo.

\begin{table}[h]
	\begin{adjustbox}{center}
		\begin{tabular}{|c|c|c|c|c|c|}
			\hline
			\rowcolor[gray]{0.8}
			\textbf{Datos} & \textbf{Algoritmo} & \multicolumn{4}{|c|}{\textbf{Validación Cruzada Repetida}} \\ \hline
			\rowcolor[gray]{0.8}			
			\multicolumn{2}{|c|}{\textbf{ }} & \textbf{Tasa Error} & \textbf{Desviación Estándar} & 			\multicolumn{2}{|c|}{\textbf{Intervalo}} \\ \hline
			
			soybean & j48 & 0,0882352941176471 & 0,033851824720493 & 0,0791987745400279 & 0,0972718136952663\\ \hline
			 & sin podar	 & 0,0948316283034953 & 0,0350435294569074 & 0,0854769910684007 & 0,10418626553859\\ \hline
			Vote	 & j48 & 0,0351083509513742 & 0,0269968624124527 & 0,0279017179932183 & 0,0423149839095301\\ \hline
			 & sin podar	 & 0,0403012684989429 & 0,0315414221939052 & 0,0318814955172923 & 0,0487210414805935\\ \hline
		\end{tabular}
	\end{adjustbox}
\end{table}

Para obtener los resultados de la tabla, utilizamos la siguientes fórmulas.

\begin{itemize}
	\item Tasa Error: \hspace{0.5cm}	 $e(h) = [\sum_{i = 1,R*k}e_i(h)] / (R*k)$
	\item Desviación Estándar: \hspace{0.5cm} $S_{e(h)} = \sqrt{1/(R*(k-1)) \quad \sum_{i = 1,R*k}{(e_i(h) - e(h))^2}} $
	\item Intervalo de confianza t-student: 	\hspace{0.5cm}	$[e(h) \pm t_{N,R*(k-1)} \quad S_{e(h)} / \sqrt{R*k}]$
	\item Siendo \hspace{0.5cm}	qt(0.95,36) = 1,688298 \hspace{0.5cm}	calculado con R
\end{itemize}
Siendo k = 10 y R = 4.


\newpage

\section{Comparativas de la estimación del error}

\subsection{Conjunto de datos - soybean}
\begin{table}[h]
	\begin{adjustbox}{center}
		\begin{tabular}{|c|c|c|c|c|c|}
			\hline
			\rowcolor[gray]{0.8}
			\textbf{Algoritmo} & \textbf{50T} & \textbf{Hold
			 Out} & \textbf{Hold Out Repetido} & \textbf{10-XV} & \textbf{4 x 10-XV}\\ \hline
			 
			 \rowcolor[gray]{0.8}
			 \textbf{J48} & \multicolumn{5}{|c|}{ } \\ \hline
			 
			 \textbf{Error} & 0,411671924290221 & 0,103004291845494 & 0,109354467203066 & 0,0821500000000001 & 0,0882352941176471 \\ \hline
			 
			 \textbf{Desviación} & 0,0195606623587688 & 0,012081503606538 & 0,0105673992752356 & 0,0106491783720623 & 0,033851824720493 \\ \hline
			 
			 \rowcolor[gray]{0.8}
			 \textbf{Sin podar} & \multicolumn{5}{|c|}{ } \\ \hline
			 
			 \textbf{Error} & 0,416403785488959 & 0,107296137339056 & 0,113711024413243 & 0,09224 & 0,0948316283034953 \\ \hline
			 
			 \textbf{Desviación} & 0,0195934860745905 & 0,0123010984913333 & 0,0206986099918583 & 0,00953312354081513 & 0,0350435294569074 \\ \hline
		\end{tabular}
	\end{adjustbox}
\end{table}

Como podemos observar en los resultados de todos los métodos para el algoritmo J48, podemos decir que parece que el mejor método de clasificación es la validación cruzada 10 particiones pues es con el que obtenemos menor tasa de error de todos, a la par con el validación cruzada repetida, con el que obtenemos resultados parecidos.
\\
\\

Con el Algoritmo unprunned, mismo racionamiento, obtenemos peores resultados que con el algoritmo J48 y decimos que el mejor método de clasificación es el 10 particiones.

\vspace{1cm}

\subsection{Conjunto de datos - vote}
\begin{table}[h]
	\begin{adjustbox}{center}
		\begin{tabular}{|c|c|c|c|c|c|}
			\hline
			\rowcolor[gray]{0.8}
			\textbf{Algoritmo} & \textbf{50T} & \textbf{Hold
			 Out} & \textbf{Hold Out Repetido} & \textbf{10-XV} & \textbf{4 x 10-XV}\\ \hline
			 
			 \rowcolor[gray]{0.8}
			 \textbf{J48} & \multicolumn{5}{|c|}{ } \\ \hline
			 
			 \textbf{Error} & 0,077720207253886 & 0,0337837837837838 & 0,0541574738003309 & 0,0342699999999999 & 0,0351083509513742 \\ \hline
			 
			 \textbf{Desviación} & 0,0136448245874235 & 0,00920790734958442 & 0,0146614150119994 & 0,0057205380477325 & 0,0269968624124527 \\ \hline
			 
			 \rowcolor[gray]{0.8}
			 \textbf{Sin podar} & \multicolumn{5}{|c|}{ } \\ \hline
			 
			 \textbf{Error} & 0,077720207253886 & 0,0337837837837838 & 0,0507790954219526 & 0,0423699999999999 & 0,0403012684989429 \\ \hline
			 
			 \textbf{Desviación} & 0,0136448245874235 & 0,00920790734958442 & 0,0118236710231369 & 0,00881980725412975 & 0,0315414221939052 \\ \hline
		\end{tabular}
	\end{adjustbox}
\end{table}

Con el conjunto de datos vote, con el algoritmo J48, parece que el método de clasificación que menor tasa de error tiene sigue siendo el validación cruzada 10 particiones, seguido del repetido y el Hold Out.
\\
\\
Para el algoritmo unprunned, el método que mejor clasifica parece ser el Hold Out, seguido del 10-XV.


\newpage

\section{Conjuntos de datos}
\begin{itemize}
	\item Soybean
		\begin{itemize}
			\item \href{https://archive.ics.uci.edu/ml/datasets/Soybean+(Large)}{https://archive.ics.uci.edu/ml/datasets/Soybean+(Large)}
			\item 683 instancias
			\item 36 atributos (35 + clase)
			\item 19 clases
		\end{itemize}	
		
	\item Vote
		\begin{itemize}
			\item \href{https://archive.ics.uci.edu/ml/datasets/congressional+voting+records}{https://archive.ics.uci.edu/ml/datasets/congressional+voting+records}
			\item 435 instancias
			\item 17 atributos (16 + clase)
			\item 2 clases
		\end{itemize}			 	
\end{itemize}

\vspace{1cm}

\section{Preguntas sobre validación cruzada}
En esta práctica, en la validación cruzada repetida, hemos
considerado como experimento base cada proceso de validación
cruzada (Método 1).
\\
Sin embargo, es más habitual considerar como experimento base
cada proceso de entrenamiento y validación sobre cada capa (fold).
(Método 2)\\
\\

\subsection{¿Qué tasa de error se obtendría con el método 2?}
Con el método 2, puede que obtengamos peores tasas de error, es decir, más altas, puesto que estamos clasificando con un conjunto de datos nuevos, no con el que hemos entrenado, por ello obtendríamos peores resultados.\\
\\
\subsection{¿Cómo espera que varíe la estimación de la varianza e intervalos de confianza con el método 2
frente al método 1?}
La estimación de la varianza será menos precisa puesto que estamos partiendo el conjunto de datos de manera que tenemos menos instancias para dicha estimación que destinamos para la clasificación.\\
En cuanto a los intervalos de confianza, mismo comportamiento, obtendríamos peores estimaciones, es decir, intervalos menos precisos, debido a la reducción de instancias en el conjunto de datos.



\end{document}
